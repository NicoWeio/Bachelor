%%%%%%%%%%%%%%%%%%%%%%%%%%%%%%%%%%%%%%%%%%%%%%%%%%%%%%%%%%%%%%%%%%%%%%%%%%%%%%%%
%%%%%%%%%%%%%%%%%%   Vorlage für eine Abschlussarbeit   %%%%%%%%%%%%%%%%%%%%%%%%
%%%%%%%%%%%%%%%%%%%%%%%%%%%%%%%%%%%%%%%%%%%%%%%%%%%%%%%%%%%%%%%%%%%%%%%%%%%%%%%%

% Erstellt von Maximilian Nöthe, <maximilian.noethe@tu-dortmund.de>
% ausgelegt für lualatex und Biblatex mit biber

% Kompilieren mit
% latexmk --lualatex --output-directory=build thesis.tex
% oder einfach mit:
% make

\documentclass[
  tucolor,       % remove for less green,
  BCOR=12mm,     % 12mm binding corrections, adjust to fit your binding
  parskip=half,  % new paragraphs start with half line vertical space
  open=any,      % chapters start on both odd and even pages
  cleardoublepage=plain,  % no header/footer on blank pages
]{tudothesis}


% Warning, if another latex run is needed
\usepackage[aux]{rerunfilecheck}

% just list chapters and sections in the toc, not subsections or smaller
% \setcounter{tocdepth}{1}
% MODIFIED: We apply this only to the rendered TOC,
% so that the TOC embedded in the PDF still contains subsections.

%------------------------------------------------------------------------------
%------------------------------ Fonts, Unicode, Language ----------------------
%------------------------------------------------------------------------------
\usepackage{fontspec}
\defaultfontfeatures{Ligatures=TeX}  % -- becomes en-dash etc.

% load english (for abstract) and ngerman language
% the main language has to come last
\usepackage[ngerman, american]{babel}

% intelligent quotation marks, language and nesting sensitive
\usepackage[autostyle]{csquotes}

% microtypographical features, makes the text look nicer on the small scale
\usepackage{microtype}

%------------------------------------------------------------------------------
%------------------------ Math Packages and settings --------------------------
%------------------------------------------------------------------------------

\usepackage{amsmath}
\usepackage{amssymb}
\usepackage{mathtools}

% Enable Unicode-Math and follow the ISO-Standards for typesetting math
\usepackage[
  math-style=ISO,
  bold-style=ISO,
  sans-style=italic,
  nabla=upright,
  partial=upright,
]{unicode-math}
\setmathfont{Latin Modern Math}

% nice, small fracs for the text with \sfrac{}{}
\usepackage{xfrac}


%------------------------------------------------------------------------------
%---------------------------- Numbers and Units -------------------------------
%------------------------------------------------------------------------------

\usepackage[
  locale=US,
  separate-uncertainty=true,
  per-mode=symbol-or-fraction,
]{siunitx}
\sisetup{math-micro=\text{µ},text-micro=µ}

%------------------------------------------------------------------------------
%-------------------------------- tables  -------------------------------------
%------------------------------------------------------------------------------

\usepackage{booktabs}       % \toprule, \midrule, \bottomrule, etc

%------------------------------------------------------------------------------
%-------------------------------- graphics -------------------------------------
%------------------------------------------------------------------------------

\usepackage{graphicx}
% currently broken
% \usepackage{grffile}

% allow figures to be placed in the running text by default:
\usepackage{scrhack}
\usepackage{float}
\floatplacement{figure}{htbp}
\floatplacement{table}{htbp}

% keep figures and tables in the section
\usepackage[section, below]{placeins}


%------------------------------------------------------------------------------
%---------------------- customize list environments ---------------------------
%------------------------------------------------------------------------------

\usepackage{enumitem}

%------------------------------------------------------------------------------
%------------------------------ Bibliographie ---------------------------------
%------------------------------------------------------------------------------

\usepackage[
  backend=biber,   % use modern biber backend
  autolang=hyphen, % load hyphenation rules for if language of bibentry is not
                   % german, has to be loaded with \setotherlanguages
                   % in the references.bib use langid={en} for english sources
]{biblatex}
\addbibresource{references.bib}  % the bib file to use
\DefineBibliographyStrings{american}{andothers = {{et\,al\adddot}}}  % replace u.a. with et al.


% Last packages, do not change order or insert new packages after these ones
\usepackage[pdfusetitle, unicode, linkbordercolor=tugreen, citebordercolor=tugreen]{hyperref}
\usepackage{bookmark}
\usepackage[shortcuts]{extdash}

%------------------------------------------------------------------------------
%------------------------------ Custom dependencies ---------------------------
%------------------------------------------------------------------------------

\usepackage{blindtext}
\usepackage{todonotes}
\usepackage[outputdir=build]{minted}
\usepackage[nameinlink]{cleveref}
\usepackage{pdfpages}

\usepackage{algorithm}
\usepackage{algpseudocode}
\renewcommand{\algorithmicrequire}{\textbf{Input:}}
\renewcommand{\algorithmicensure}{\textbf{Output:}}

%------------------------------------------------------------------------------
%------------------------------ Custom commands -------------------------------
%------------------------------------------------------------------------------

\newcommand{\corn}{\textsc{Corn}}
\newcommand{\dsea}{\textsc{Dsea}}
\newcommand{\dseaplus}{{$\text{\textsc{Dsea}}^+$}} % TODO: messes up spacing
\newcommand{\icecube}{IceCube}

\newcommand{\icecubeneutrinoobservatory}{\icecube Neutrino Observatory}

% Wikipedia-style "citation needed" macro
% https://gist.github.com/martinarroyo/b9e0a963ad27169a6eee?permalink_comment_id=2355734#gistcomment-2355734
% \newcommand{\citationneeded}[1][]{\textsuperscript{\color{blue} [citation needed: #1]}}
\newcommand{\citationneeded}{\textsuperscript{\color{blue} [citation needed]}}

%------------------------------------------------------------------------------
%-------------------------    Angaben zur Arbeit   ----------------------------
%------------------------------------------------------------------------------

\author{Nicolai Weitkemper}
\title{Ordinal classification with neural networks in DSEA}
\date{2022}
\birthplace{Soest}
\chair{Lehrstuhl für Experimentelle Physik V}
\division{Fakultät Physik}
\thesisclass{Bachelor of Science}
\submissiondate{30. September 2022}
\firstcorrector{Prof.~Dr.~Dr.~Wolfgang~Rhode}
\secondcorrector{Prof.~Dr.~Johannes~Albrecht}

% tu logo on top of the titlepage
\titlehead{\includegraphics[height=1.5cm]{logos/tu-logo.pdf}}

\begin{document}
\frontmatter
\maketitle

% Gutachterseite
\makecorrectorpage

% hier beginnt der Vorspann, nummeriert in römischen Zahlen
\thispagestyle{plain}

\section*{Abstract}
% - Neutrinos
% - IceCube
% - DSEA
% - CORN
% - THIS THESIS

The deconvolution of energy spectra is a [common] problem in neutrino astronomy.
As an inverse problem,
  it requires special methods to solve.
The \textbf{D}ortmund \textbf{S}pectrum \textbf{E}stimation \textbf{A}lgorithm (\dsea{}) \cite{dsea_unification}
solves the deconvolution problem
  by using a classifier to estimate the discretized energy spectrum
  and iteratively reweighting the training data.
%
In the given problem,
  neutrino energies are \emph{ordinal} quantities,
    which means that the order of the energies is important. % COPILOT
However,
  most classifiers do not respect this property.
%
% NOTE: Duplicate of introduction.tex
Previous works
have focused either
  on respecting ordinality \cite{dsea_jan}
  or on using a neural network as a classifier \cite{dsea_samuel}.
This thesis aims to combine the advantages of both approaches:
  the flexibility of neural networks
  and the potential improvements in physical plausibility
    due to respecting ordinality.
The \corn{} framework,
  which [provides] ordinal neural networks,
is adapted to work with \dsea{}
and optimized and evaluated on simulated data from \icecube{}.



\section*{Kurzfassung}
\begin{foreignlanguage}{german}
Hier steht eine Kurzfassung der Arbeit in deutscher Sprache inklusive der Zusammenfassung der
Ergebnisse.
Zusammen mit der englischen Zusammenfassung muss sie auf diese Seite passen.

\blindtext[1]
\end{foreignlanguage}

\setcounter{tocdepth}{1}
\tableofcontents
\setcounter{tocdepth}{2}

\mainmatter
% Hier beginnt der Inhalt mit Seite 1 in arabischen Ziffern
\chapter{Introduction}
% █ astroparticle physics
Astroparticle physics is a relatively new field of physics
  that explores the universe
    with messenger particles.
They carry information about
    the production processes
    and the environment they were created in,
  which can be obtained upon their detection.
% This way, AGN, supernovae, and other astrophysical phenomena can be studied.
% █ neutrino astronomy
Neutrinos are especially valuable messenger particles
  because their propagation path is not affected by electromagnetic fields,
  and they can propagate long distances without interacting with matter.
% Therefore,
%   both their source and their energy can be determined
%   by measuring the direction and energy of
%   decay products that they produce in the detector on Earth.

% The most common messenger particles are neutrinos,
% Many high-energy processes are assumed to produce neutrinos.

% █ IceCube
\icecube{} is a detector for such neutrinos,
  which is located at the South Pole.
The Antarctic ice is used as detector material
  and is sensitive to high-energy neutrinos. % COULDDO: imprecise
One goal of analyses with \icecube{} is
  to obtain a neutrino energy spectrum.

% █ DSEA
The \acf{DSEAPLUS} \cite{dsea_unification}
is a method to reconstruct the (discretized) neutrino energy spectrum
  from measured data of \icecube{}.
While classifiers such as random forests
have been successfully applied to \icecube{} data using \dsea{} \cite{hymon2021seasonal},
the possibilities of neural networks and deep learning are still largely unexplored. % Naja…
%
\citeauthor{dsea_samuel} \cite{dsea_samuel}
combined \dsea{} with a neural network for the first time.
%
\citeauthor{dsea_jan} \cite{dsea_jan},
  on the other hand,
focused on the ordinality of the discretized energy spectrum
  by employing \emph{LogisticAT} \cite{logisticat},
    a classifier
      that considers
        the ordering of energy bins
      in its loss function.
%
% COULDDO: Differentiate from abstract ↓
This thesis aims to combine the advantages of both approaches:
  the flexibility of neural networks
  and the potential improvements in physical plausibility
    due to respecting ordinality.
% █ CORN
The \ac{CORN} framework \cite{corn}
  allows neural networks to respect ordinality
  by using a special
    loss function
    and activation function
      in the output layer.
It is adapted to work with \dsea{} by
  converting from conditional confidences to per-class confidences
  and adding support for sample weights.
This combination of \ac{CORN} and \dsea{}
is then optimized and evaluated
  on simulated data from \icecube{}.

% █ structure
This paper is organized as follows:
\Cref{sec:neutrino_astronomy} briefly introduces \icecube{} and important concepts of neutrino astronomy.
\Cref{sec:dsea} describes the deconvolution problem and the \dsea{} approach to solving it.
\Cref{sec:ordinal} explains the benefits of ordinal classification and introduces the \ac{CORN} framework.
\Cref{sec:unfolding} describes the setup for hyperparameter searches and evaluates the performance of the optimized model.
\Cref{sec:summary} summarizes the thesis and discusses future work.

\chapter{Neutrino Astronomy}
  \section{Neutrinos}

% - What are neutrinos?
Neutrinos are elementary particles with no electric charge and very little mass.
Their low interactivity
  makes them difficult to detect,
  but is also the reason why they are valuable messenger particles for cosmology:
Because they are not impacted by the electromagnetic force,
% and very little by gravity,
  their direction of travel is not affected by cosmic magnetic fields.
Neither does the interstellar medium absorb them in significant amounts.
Therefore,
  their direction of travel can be used to trace back the path of the matter that produced them.
Together with the energy of the neutrino,
  this information can be used to determine the properties of the matter that produced them.
This is the basis of neutrino astronomy.

% - What are their properties?

\subsection{Sources}
% - What are their sources?
%   - Nuclear reactors
%   - Solar
%   - AGNs
%   - Supernovae
%   - Atmospheric neutrinos
Neutrinos have various astronomical sources,
% TODO: We are not sure about these ↓
including
  supernovae,
  pulsars, % Oxford comma
  and active galactic nuclei.
Neutrinos are also produced in
  the Sun
  and in the Earth's atmosphere
  as well as in nuclear reactors.
They cover a wide range of energies, from a few MeV to a few TeV,
  depending on the source.

\begin{figure}
  \centering
  \includegraphics[width=0.75\textwidth]{content/plots/halftime/neutrinos-energy.png}
  \caption{Neutrino flux as a function of their energy.}
  \label{fig:neutrinos:flux_spectrum}
\end{figure}


\subsection{Neutrino oscillations}
While current models of neutrino sources predict a ratio of
  $\nu_e:\nu_\mu:\nu_\tau = 1:2:0$,
the observed ratio on Earth is
  $\nu_e:\nu_\mu:\nu_\tau = 1:1:1$.
This discrepancy is explained by neutrino oscillations \cite{neutrinos_beacom},
  which are a consequence of the fact that neutrinos have mass.
The mass of the neutrinos is very small,
  but it is large enough to cause oscillations between the different flavors,
  given the large distances that cosmic neutrinos travel.


% - What are their interactions?
%   - Weak interaction
\subsection{Interaction with matter}
  \todo{Move to IceCube?}
Neutrinos interact with matter via the weak interaction.
In order to compensate for the low cross-section of the weak interaction,
  the effective detector volume is maximized by utilizing existing naturally occurring detector materials,
  such as
    the Earth's atmosphere,
    the sea,
    or the ice in the Antarctic.

% - How are they detected?
%   - → IceCube, later

  \section{IceCube}
\blindtext[5]

\begin{figure}
  \centering
  \includegraphics[width=\textwidth]{content/img/icecube_detector_schematic.jpg}
  \caption{
    Schematic presentation of the IceCube Neutrino Observatory. \cite{icecube_homepage}
    % TODO: longer description
  }
  \label{fig:img:icecube}
\end{figure}

\chapter{Solving the deconvolution problem with DSEA}
  \section{The deconvolution problem} % / Inverse problems
% \subsection{Inverse problems}
% TODO: Make clear the difference between inverse problems and deconvolution
Inverse problems are omnipresent in modern physics.
They occur whenever a physical quantity is measured indirectly.
For example,
the intensity of light can measured by a photodetector,
  which converts the light into an electrical signal,
    thus measuring the intensity \emph{indirectly}.
However,
this conversion is not perfect:
The detector has limited acceptance and resolution,
and the signal is subject to noise.
% forward / inverse problem
For neutrino astronomy,
  energy measurements are even more indirect,
    going from neutrinos to leptons to Cherenkov light to electrical signals.
%
The deconvolution problem is to reconstruct the physical quantity of interest
  from the indirect measurements.

% - Fredholm integral equation
Mathematically,
the deconvolution problem can be formulated as an integral equation.
%
The notation is based on \citeauthor{deconvolution_blobel} \cite{deconvolution_blobel}.
% In particular,
%   the meaning of $x$ and $y$ is swapped in the other chapters.
A set of single physical quantities $x$
  with $a \leq x \leq b$ % TODO: Style
  (such as the energy of a neutrino)
can be interpreted as a sample from a probability density $f(x)$.
%
Given
  the true distribution of a physical quantity $f(x)$,
  the measured distribution $g(x)$,
  and a \emph{response function} $A(x, y)$,
  % TODO: mention \Omega
% the deconvolution problem is
%   to find $f(x)$
%     such that $g(x) = A(x, y) f(y)$.
the deconvolution problem is given by
the Fredholm integral equation of the first kind: \cite{deconvolution_blobel}
\begin{equation}
  \label{eq:deconvolution_problem:fredholm}
  % \int_{\mathbb{R}^d} \phi(x) \, \varphi(x - y) \, \mathrm{d}x = \varphi(y) \, .
  \int_a^b A(x, y) f(x) \, \mathrm{d}x = g(y) \, .
\end{equation}


\subsection{Discretization}
In the context of physical measurements,
the integral equation is discretized
  to account for the finite number of samples.
The continuous distribution functions $f(x)$ and $g(x)$ are replaced by vectors $\vec{f}$ and $\vec{g}$,
and the kernel (response function) $A(x, y)$ by a \emph{transfer matrix} $\symbf{A}$.
%
% Disregarding the bias term, % TODO: if there was one in the first place…
The discretized deconvolution problem is then given by
\begin{equation}
  \label{eq:deconvolution_problem:discretized}
  \symbf{A} \vec{f} = \vec{g} \, .
\end{equation}

In practice,
the transfer matrix $A(x, y)$ can be approximated
by Monte Carlo simulations of the detector,
  where both the true and the measured quantities are known.
Given an actual set of measurements $\vec{g}$,
the deconvolution problem can then be solved
by inverting the transfer matrix $\symbf{A}$:
\begin{equation}
  \label{eq:deconvolution_problem:discretized:inverse}
  \vec{f} = \symbf{A}^{-1} \vec{g} \, .
\end{equation}

However,
% as is common with inverse problems,
the matrix inversion operation is usually \emph{ill-conditioned},
  leading to numerical instabilities
  and oscillations in the solution.

% It is \emph{ill-conditioned}:
%   the solution can change drastically
%     when the measurements are slightly perturbed.
% % COPILOT: This is because the solution is not unique:
% %   there are many different physical quantities
% %     that can produce the same measurements.

One approach to overcome this problem
is \emph{regularization}.
% Regularization is a technique
% to make the deconvolution problem well-posed
% by adding a penalty term to the objective function.
Regularization allows for better results
  at the cost of introducing additional assumptions and parameters
    (\emph{Bias-variance trade-off}).
A common regularization technique
  is to penalize the second derivative of the solution.
    % which is known as \emph{Tikhonov regularization}. % TODO: Is it really?
It incentivizes the solution to be smooth,
  which is often a reasonable assumption for distributions of physical quantities.


  \section{Dortmund Spectrum Estimation Algorithm} \label{sec:dsea:dsea}
The \textbf{D}ortmund \textbf{S}pectrum \textbf{E}stimation \textbf{A}lgorithm
  (\dsea{})
  \cite{dsea_unification}
is an iterative method for solving the previously stated deconvolution problem.
The extended version \cite{dsea_mirko}
  that
    % corrects the re-weighting of training examples and
    gives more control over the speed of convergence
is also referred to as \dseaplus{}.
%
A formal definition of the \dseaplus{} algorithm is given in \autoref{sec:alg:dseaplus}.


\subsection{Deconvolution as a classification task} % identical to Jan's paper
\dsea{} makes use of classifiers to solve the deconvolution problem.
This requires the deconvolution problem to be
  discretized (see \autoref{sec:dsea:deconvolution_problem:discretization})
  and reformulated as a multinomial classification task.

Any classifier that outputs probabilities for each class
  can be used with \dsea{}.
This is an advantage,
  as the choice of a classifier can be tailored to the specific problem at hand.
% ORIG: The novel algorithmic framework Dsea is unique among these algorithms,
% because it translates the deconvolution problem into a multinomial classification task,
% thus opening deconvolution to the field of machine learning and the rapid advances being made in that field.
Contrary to other algorithms like
  \textsc{Truee} / \textsc{Run} \cite{milke2013} or
  \textsc{Ibu} \cite{dagostini1995, dagostini2010},
no restrictions on the input data are imposed.

Furthermore,
\dsea{} transparently provides the contributions of individual observations to the deconvolved spectrum.
This not only gives deeper insight into the performance of the algorithm,
but also allows for time-dependent deconvolution \cite{dsea_mirko}. % NOTE: … for example


\clearpage % guidance only
\subsection{Procedure}

% The \dsea{} algorithm is an iterative procedure.
% $N$ is the number of bins or classes in the spectrum.
% % $\hat \mathbf{f}_j^{(0)}$ is the initial guess for the $j$-th bin of the deconvolved spectrum.
% $\hat \mathbf{f}_j^{(k)}$ is the $j$-th bin of the deconvolved spectrum after $k$ iterations.


\subsubsection{Initialization}
Since no prior knowledge about the true spectrum is available,
  the initial spectrum is chosen to be uniform.
Given $I$ bins,
  each bin is initialized to
\begin{equation}
  % NOTE: Mirko made f bold, but not p. Jan and Samuel didn't make anything bold.
  % I'll follow Mirko's lead, for no particular reason.
  \hat \mathbf{f}_i^{(0)} = \frac{1}{I} \quad \forall i \, .
\end{equation}

% TODO: Explain variables here, not below?
The initial weights are then determined as in \autoref{eqn:dsea:weighting}.

% █ The resulting bins are used as classes.


\subsubsection{Iteration}
First,
the classifier is trained
  on the training data
to predict the class of one sample at a time,
where each sample is weighted according to its true class.

Second,
the classifier is used to predict the class of each sample
in the observed~(/test) data.

Third,
the predicted classes are used to
  % reweight the training samples
  get an updated estimate of the spectrum
  and,
    consequently,
  updated weights for the next iteration.
%
The new spectrum is determined by the sum of the confidences of all events.
% NOTE: i (DSEA) ≙ q (CORN)
For each energy bin with index~$i$,
this can be written as
\begin{equation}
  \hat \mathbf{f}_i = \frac{1}{N} \sum_{n=1}^N \hat c_{i,n} \, ,
\end{equation}
where $N$ is the number of events in the observation dataset,
$k$ is the current iteration number, % Oxford comma
and $\hat c_{i,n}$ is the confidence
  that event $n$ belongs to class $i$.
The factor $\sfrac{1}{N}$ is introduced to normalize the spectrum
to a true probability density distribution.
%
% █ The resulting spectrum is the preliminary deconvolution result.
The weights of the training samples are then updated according to the new spectrum~$\hat \mathbf{f}_i^{(k)}$. % …preliminary deconvolution result.
In \dseaplus{}, the reconstructed spectrum is divided by the training spectrum % NOTE: → fixweighting
  in order to mitigate the impact of the training spectrum on the deconvolution result.
A more detailed reasoning is given in \cite{dsea_mirko}.
The updated weights are given by
\begin{equation}
  \label{eqn:dsea:weighting}
  w_i^{(k+1)} = \frac{\hat \mathbf{f}_i^{(k)}}{\mathbf{f}_i^\text{train}} \, ,
\end{equation}
where $w_i^{(k+1)}$ is the weight applied to training samples with true~bin~$i$ in iteration~$k+1$, % (Oxford comma)
% $\hat \mathbf{f}_i^{(k)}$ is
%   the value of the $i$-th bin in
%   the current deconvolution result, % Oxford comma
and $\mathbf{f}_i^\text{train}$ is
  the value of the $i$-th bin in
  the training spectrum.


The iterative procedure is repeated
  with the updated weights
until
  convergence
    % (determined by the $\chi^2$ distance between the current and previous deconvolution result)
  or,
    in case of fixed step sizes,
  a maximum number of iterations
is reached.
%
%
% \subsubsection{Result}
% TODO: Explain convergence…
The final deconvolution result is the spectrum obtained in the last iteration.


\clearpage % guidance only
\subsection{Step size functions} \label{sec:dsea:dsea:stepsize}
\dseaplus{} introduces the concept of a step size $\alpha$,
which allows the user to control the speed of convergence,
which in turn has a significant impact on the quality of the result.

A \emph{step} $p_i^{(k)}$ is the difference between the current and previous deconvolution result:
\begin{equation}
  p_i^{(k)} = \hat \mathbf{f}_i^{(k)} - \hat \mathbf{f}_i^{(k-1)} \, .
\end{equation}
Instead of updating the spectrum with the current deconvolution result $\mathbf{f}_i^{(k)}$ directly,
the step is multiplied with the step size
and added to the previous deconvolution result:
\begin{equation}
  \hat \mathbf{f}_i^{(k)+} = \hat \mathbf{f}_i^{(k-1)} + \alpha \cdot p_i^{(k)} \, .
\end{equation}
This improved estimate $\hat \mathbf{f}_i^{(k)+}$ is then considered instead of $\hat \mathbf{f}_i^{(k)}$.

While the original \dsea{} algorithm uses a fixed step size of $\alpha = 1$,
\dseaplus{} allows arbitrary constants $\alpha > 0$
or functions of the iteration number $k$.
Commonly used step size functions include
multiplicative decay
  $\alpha^{(k)} = k^{\eta - 1}$
and exponential decay
  $\alpha^{(k)} = \eta^{(k - 1)}$,
each with a \emph{decay rate} $0 < \eta < 1$.

These decaying step sizes ensure that the algorithm converges,
decreasing the importance of the maximum number of iterations $K$,
while enabling the use of a stopping criterion,
  expressed by a minimum $\chi^2$~distance $\epsilon$ between iterations.
% NOTE: This had already been suggested in the original DSEA paper (→ dsea_tim).


The utilization of \emph{adaptive step sizes} \cite{dsea_mirko} can further improve the convergence of the algorithm
by choosing an optimal step size $\alpha$ for each iteration.
This is achieved by searching
  along the direction of the step % aka. search direction
  for the optimal step size $\alpha$.
    % which minimizes an objective function…
% TODO ↓
% The method provided by \cite{dsea_mirko} discretizes the training data u
%   sing a decision tree,
%   thus adding a hyperparameter $J$
%     that controls the number of leaves in the tree.

\chapter{Ordinal Classification}
  \section{On nominal and ordinal data}
  \blindtext[3]
  % LogisticAT ?
  \section{CORN} \label{sec:ordinal:corn}
% Conditional Ordinal Regression for Neural Networks
% NOTE: Ordinal Regression = Ordinal Classification. It's an intermediate problem between classification and regression. (→ corn)
\acf{CORN} \cite{corn}
is a framework for ordinal classification in neural networks.
It is based on the ideas of binary subtasks and conditional probabilities
and improves upon
  its direct predecessor \ac{CORAL} \cite{coral}
  as well as the approach by \citeauthor{extended_binary_nn} \cite{extended_binary_nn}.
%
% NOTE: - Niu et al.'s approach has rank inconsistency
%       - CORAL has rank consistency,
%           but is limited in expressiveness due to its weight sharing.
%
% ORIG: Instead, CORN uses a new training procedure with conditional training subsets
% that ensures rank consistency through applying the chain rule of probability.
%
% COULDDO: It might make sense to explain the flaws of the previous approaches here,
% but that would also require explaining rank consistency,


\subsection{Method} \label{sec:corn:method}
% COULDDO: This subsection heavily borrows from the CORN paper.
% Although it's cited, it could be rephrased some more.
% @Leonora thinks it's fine as is.
%
Let $D = \left\{ \mathbf{x}^{[i]}, y^{[i]} \right\}_{i=1}^N$ denote a data set of $N$ training examples,
where $\mathbf{x}^{[i]}$ is the $i$-th example and $y^{[i]}$ is its class label.
Since the class labels are ordinal, they are referred to as \emph{rank} labels.
Each rank label is an element of the set of all ranks $\{r_1, r_2, \ldots, r_K\}$,
  where
  $K$ is the number of ranks
  and $r_1 < r_2 < \ldots < r_K$.

For every rank label $y^{[i]}$,
$K - 1$ subtasks are created.
Each subtask $y^{[i]}_k \in \{0, 1\}$ is a binary classification task,
  where
    $y^{[i]}_k = 1$ if $y^{[i]} > r_k$ (in words: {$y^{[i]}$ exceeds rank $r_k$})
    and $y^{[i]}_k = 0$ otherwise.
This method of creating binary subtasks is referred to as \emph{extended binary classification} \cite{extended_binary}.
% …and predates CORN.

Given a test example $\mathbf{x}^{[i]}$
and probability predictions $f_k(\mathbf{x}^{[i]}) \in [0,1]$ for each subtask $k$,
the \emph{rank index} $q \in \{1, 2, \ldots, K\}$ is computed as
\begin{equation}
  q = \sum_{k=1}^{K-1} \mathbb{1}\left\{f_k(\mathbf{x}^{[i]}) > 0.5\right\} \ .
\end{equation}
The predicted rank label is then obtained via
$h(\mathbf{x}^{[i]}) = r_q$,
where $h: \mathcal{X} \to \mathcal{Y}$ is the mapping from input space $\mathcal{X}$ to output space $\mathcal{Y}$ %,
which minimizes the CORN loss function.


\emph{Rank monotony} describes a desirable property of the rank labels,
whereby the probability of exceeding a rank $r_k$ is always greater than or equal to the probability of exceeding a higher rank $r_{k+1}$.
% a rank label being exceeded by a higher rank label is always higher than the probability of being exceeded by a lower rank label.
%
While not strictly necessary for the computation of rank indices $q$ (or ranks $r_q$),
it is intuitively clear that imposing such a restriction could improve the quality of predictions.
%
\Ac{CORN} ensures rank monotony
  $f_1(\mathbf{x}^{[i]}) \leq f_2(\mathbf{x}^{[i]}) \leq \ldots \leq f_{K-1}(\mathbf{x}^{[i]})$
by applying the chain rule of probability
\begin{equation}
  \hat{P}\left(y^{[i]} > r_k\right) = \prod_{j=1}^k f_j(\mathbf{x}^{[i]})
\end{equation}
to the conditional probabilities
% COULDDO: It should be noted at the first mention (above) that these are conditional probabilities.
\begin{equation}
  f_k(\mathbf{x}^{[i]}) = \hat{P}\left( y^{[i]} > r_k \mid y^{[i]} > r_{k-1} \right) \ .
  \label{eq:corn:conditional_probabilities}
\end{equation}
%
% COULDDO: Add a TikZ visualization similar to Fig. 1 in the CORN paper | see content/tikz/rank_consistency.tex

\Ac{CORN} also provides the loss function.
The conditional nature of the predictions $f_k(\mathbf{x}^{[i]})$
  (see \autoref{eq:corn:conditional_probabilities})
is respected
by splitting the training data into conditional training subsets $S_k$
  each time the loss function is computed:
\begin{align*}
  S_1 &: \text{all } \left\{\left( \mathbf{x}^{[i]}, y^{[i]} \right)\right\} \, \text{ for } i \in \{1, \ldots, N\} \, , \\
  S_2 &: \left\{\left( \mathbf{x}^{[i]}, y^{[i]} \right) \mid y^{[i]} > r_1 \right\} \, , \\
  &\cdots \\
  % NOTE: I made `K-2` uppercase, because I think it's lowercase by mistake in the paper.
  S_{K-1} &: \left\{\left( \mathbf{x}^{[i]}, y^{[i]} \right) \mid y^{[i]} > r_{K-2} \right\} \, .
\end{align*}
In words,
for $k > 1$,
  $S_k$ contains all training examples
  for which the rank label exceeds rank $r_{k-1}$
    (or, equivalently, for which the observed probability of exceeding rank $r_{k-1}$ is $1$).
The first subset $S_1$ contains all training examples.

The definition of the loss function is shown in \autoref{eqn:corn:loss}.
Here, $|S_k|$ denotes the number of elements in the subset $S_k$.
\begin{multline}
  \label{eqn:corn:loss}
  L(\mathbf{X}, \mathbf{y}) =
  - \frac{1}{\sum_{j=1}^{K-1} |S_j|}
  \sum_{j=1}^{K-1}
  \sum_{i=1}^{|S_j|}
  \Bigl[
    \log(f_j(\mathbf{x}^{[i]})) · \mathbb{1}\left\{y^{[i]} > r_j\right\}
    \\
    +
    \log(1 - f_j(\mathbf{x}^{[i]})) · \mathbb{1}\left\{y^{[i]} \leq r_j\right\}
  \Bigr]
\end{multline}


\subsection{Obtaining Probabilities from CORN} \label{sec:ordinal:corn:probas}
% OR: Converting threshold to per-class probabilities
As explained in \autoref{sec:corn:method},
% CORN returns conditional (?) probabilities based on the binary classification subtasks it uses.
\Ac{CORN} only returns a single predicted rank $r_q$ for a given input $\mathbf{x}$.
% … and internally has no notion of “absolute” per-class probabilities.
% NOTE: Don't get confused here:
% - the output layer represents the conditional probabilities of exceeding a rank label
% - the chain rule of probability is used to ensure rank monotony and converts the conditional probabilities into probabilities of exceeding a rank label
% - the rank index $q$ is computed from the probabilities of exceeding a rank label
In contrast,
\dsea{} is strongly coupled to the idea of per-class probabilities (confidences).
Therefore, a conversion is necessary.
Under the assumption that all energies belong to one of the ranks (bins),
such a conversion is realizable.

As an example, given four rank indices $q \in \{0, 1, 2, 3\}$,
the conditional probabilities are
\newcommand{\myP}{\hat{P}}
\begin{align*}
  % NOTE: P[q>(-1)] = 1 and P[q>3] = 0 :)
  \myP(q=0) &= 1 - \myP(q>0) \\
  \myP(q=1) &= \myP(q>1) - \myP(q>1) \\
  \myP(q=2) &= \myP(q>1) - \myP(q>2) \\
  \myP(q=3) &= \myP(q>2)
\end{align*}

% COULDDO: Explain in layman's terms (@Leonora)
A more detailed explanation is given in \autoref{sec:appendix:corn_probas}.


% \subsection{…}
% COULDDO: This does not belong to the subsection "Getting probabilities from CORN", but isn't really a subsection on its own either.
% In order to support the weighting of individual samples,
% the CORN loss function is modified
% to include the weight as a factor.
% The updated code is shown in \autoref{sec:appendix:corn_weighting}.
% NOTE: I followed @Karolins suggestion to remove the code completely.

\chapter{Unfolding with CORN}
  \section{Setup}
% TODO: Write a short introduction to the setup

\subsection{Monte Carlo Dataset}
The unmodified dataset \cite{icecube_mc} consists of about 13 million Monte Carlo simulated \emph{upgoing} neutrino events,
% NOTE: Exact number is 13336413 before any preprocessing.
with energies ranging from \SI{E2}{\giga\electronvolt} to \SI{E8}{\giga\electronvolt}.
% TODO: Explain “upgoing”.
The energies are stored under the key \texttt{MCPrimary.energy}. % TODO: Denglisch?
See \autoref{fig:dataset:raw:histogram} for a histogram of the data.
% TODO: Number of features: 79 (Jan) or 99-1 (my notebook)?
% TODO: The smallest bin contains about XY events.

To ensure comparability to \cite{dsea_samuel} and \cite{dsea_jan},
only the first \num{500000} events of the dataset are considered.
This also allows for more thorough hyperparameter optimization,
which would otherwise be limited by the available computational resources as well as the timeframe of this thesis.

Unless otherwise stated, % TODO: Do I ever do this?
\SI{90}{\percent} of the data is used for training,
while the remaining \SI{10}{\percent} is used for evaluation.
% No separate validation set is used.


\subsection{Feature selection}
Since
  unfolding is highly dependent on the selection of features \citationneeded{}
  and computational resources are limited,
not all available features are used.

\citeauthor{dsea_jan} has employed the \emph{mRMR} (Minimum Redundancy Maximum Relevance) feature selection algorithm \cite{mrmr}
to select the 12 most relevant features \cite{dsea_jan}.
The algorithm takes into account both
  the \emph{relevance} of a feature to the target variable,
    measured by their correlation,
  and the \emph{redundancy} of a feature to other features.
This way,
the \emph{minimal-optimal} set of features is selected,
  in contrast to the \emph{all-relevant} set of features,
    which would also contain redundant features.
    % which would be selected by a simple relevance-based feature selection algorithm. % Copilot blah blah
A list of said features is provided in \autoref{tab:features_best}.
They are re-used in this thesis.

\begin{table}
    \centering
    \caption{
      12 best features according to the \emph{mRMR} algorithm \cite{dsea_jan}.
    }
    \label{tab:features_best}
    \begin{tabular}{l}
        \toprule
        % \midrule
        % NOTE: MCPrimary.energy is excluded
        \texttt{SplineMPEDirectHitsICE.n\_dir\_doms} \\
        \texttt{VariousVariables.Cone\_Angle} \\
        \texttt{SplineMPECramerRaoParams.variance\_theta} \\
        \texttt{Borderness.Q\_ratio\_in\_border} \\
        \texttt{SplineMPETruncatedEnergy\_SPICEMie\_BINS\_MuEres.value} \\
        \texttt{SplineMPETruncatedEnergy\_SPICEMie\_DOMS\_Neutrino.energy} \\
        \texttt{SplineMPEDirectHitsICB.n\_late\_doms} \\
        \texttt{Dustyness.n\_doms\_in\_dust} \\
        \texttt{LineFitGeoSplit1Params.n\_hits} \\
        \texttt{SplineMPEDirectHitsICC.dir\_track\_hit\_distribution\_smoothness} \\
        \texttt{SPEFit2GeoSplit1BayesianFitParams.logl} \\
        \texttt{SplineMPECharacteristicsIC.avg\_dom\_dist\_q\_tot\_dom} \\
        \bottomrule
    \end{tabular}
\end{table}


\subsection{Transformation}
It has been shown that none of the selected features are normally distributed \cite{dsea_jan}.
In accordance with \cite{dsea_jan},
the features are therefore transformed using the \emph{Yeo-Johnson} transformation \cite{yeo_johnson},
a power transformation which reduces skewness.
% https://en.wikipedia.org/wiki/Power_transform#Yeo–Johnson_transformation
% TODO: Verify positive effect on performance
% TODO: Add formula

Additionally, zero-mean, unit-variance normalization is applied to all features.


\subsection{Discretization}
As described in \autoref{sec:dsea:dsea}, \dsea{} requires discrete energy classes.
The target variable \texttt{MCPrimary.energy} is therefore discretized into \num{10} bins
(in accordance with \cite{dsea_samuel}).
%
Contrary to \cite{dsea_jan} and \cite{dsea_samuel},
under- and overflow bins are added
  in order to allow for the application to real data.
%
The lower limit of the overflow bin is chosen so that it contains a similar number of events as the previous bin,
ensuring sufficient statistics.
A lower energy limit of \SI{E5}{\giga\electronvolt} (in accordance with \cite{dsea_samuel}) was found to satisfy this requirement.
%
The underflow bin is assigned to the energy range from \SI{E2}{\giga\electronvolt} to $10^{2.1} \si{\giga\electronvolt}$ % TODO: language
% \qtyrange{E2}{E2.1}{\giga\electronvolt}.
  because the dataset does not contain any events with exceptionally low energies below \SI{E2}{\giga\electronvolt}.
Again, the event count in the underflow bin is chosen to be similar to the neighboring bin.
%
The remaining \num{8} bins are spaced logarithmically between the under- and overflow bins
  so that the entire energy range of the Monte Carlo dataset is covered.

A histogram utilizing the aforementioned bins is shown in \autoref{fig:dataset:discretized:histogram}.

\begin{figure}
  \centering
  \includegraphics[scale=1]{content/plots/dataset_500k:discretized:histogram_full.pdf}
  \caption{Energy spectrum of the 500k Monte Carlo dataset using the discretized energy ranges as bins.}
  \label{fig:dataset:discretized:histogram}
\end{figure}


\subsection{Neural network}
A PyTorch \cite{pytorch} implementation of the \corn{} method
as well as several examples demonstrating its application on different datasets
is provided by \cite{corn}.
% TODO: Reference their tabular data / cement example?
%
This work makes use of said implementation of \corn{}
and hence the PyTorch framework.
Additionally,
  PyTorch Lightning \cite{pytorch_lightning},
  TorchMetrics \cite{torch_metrics}, % Oxford comma
  and scikit-learn \cite{sklearn}
  are used.

The neural network consists of \num{4} fully connected hidden layers.
The input layer has \num{12} neurons,
  corresponding to the number of features,
while the output layer has \num{9} neurons,
  corresponding to the number of binary classification subtasks,
    i.e. the number of bins minus one.
% \corn in the output layer…
% In total, the neural network has \num{TODO} neurons.

The number of neurons in the hidden layers is shown in \autoref{tab:nn_shape}.
\todo{Write a few sentences about these bullet points.}
\begin{itemize}
  \item Leaky ReLU activation function
  \item Fully connected layers
  \item Adam optimizer
\end{itemize}

\begin{table}
  \centering
  \begin{tabular}{S[table-format=3.0] c}
    \toprule
    {neurons} & {activation function} \\
    \midrule
    12  & – \\
    120 & leaky ReLU \\
    240 & leaky ReLU \\
    120 & leaky ReLU \\
    12  & leaky ReLU \\
    9   & leaky ReLU \\
    \bottomrule
  \end{tabular}
  % TODO: Ich habe immer gelernt: „Abbildungen haben eine Unterschrift, Tabellen eine Überschrift.“ @Karolin widerspricht dem scheinbar.
  \caption{
    Shape and activation functions of the neural network.
    The number of neurons in the input and output layers is determined by the number of features and bins, respectively.
  }
  \label{tab:nn_shape}
\end{table}

% TODO: Reference \corn loss function?

\textsc{Adam} (Adaptive Moment Estimation) \cite{adam} is used as the optimizer.
% It combines the benefits of both AdaGrad and RMSProp.

The neural network keeps its weights between \dsea{} iterations.
This was found to have no significant effect on the performance \cite{dsea_samuel}. % (\texttt{one\_model})


\subsection{DSEA}
For this work, the Python implementation of \dsea{} \cite{dsea_code} by \citeauthor{dsea_mirko} is used.
It expects a \emph{scikit-learn} classifier.
In order to interface with this library,
a wrapper class is implemented,
  which exposes a constructor as well as the needed methods
  \mintinline{python}{fit(X, y, sample_weight)} and
  \mintinline{python}{predict_proba(X)}.

  \section{Performance metrics}
In order to evaluate the performance of the models
and to compare them to prior works,
several metrics are used.
%
For the calculation of all metrics that are mentioned here are,
scikit-learn \cite{sklearn} is used.


\subsection{Accuracy} \label{sec:unfolding:metrics:accuracy}
The \emph{accuracy} \cite{accuracy} is the fraction of correctly classified events to the total number of events.
It is a common metric for classification tasks,
but it is not ideal for ordinal classification
  since it does not take the ordering of the classes into account.
For example,
the metric is the same for
a misclassification by one rank
and a misclassification by two ranks.
%
Nonetheless,
it gives an indication of the overall performance of the model.


\subsection{Mean absolute error} \label{sec:unfolding:metrics:mae}
The \emph{mean absolute error} (\textsc{Mae}) \cite{mae} is a metric that is commonly used for regression tasks. % TODO
It is defined as
\begin{equation}
  \text{\textsc{Mae}} = \frac{1}{N} \sum_{i=1}^N \left| y^{[i]} - \hat{y}^{[i]} \right|
\end{equation}
where $N$ is the number of events,
$y^{[i]}$ is the true value of the $i$-th event,
and $\hat{y}^{[i]}$ is the predicted value of the $i$-th event. % Oxford comma

% Because of its simple definition,
% it can be easily understood and interpreted.

Since the absolute value of the error is considered,
overestimation and underestimation are treated equally
and do not cancel each other out.
% NOTE: The referenced paper gives some good arguments for using MAE over RMSE.
In contrast to the \emph{root mean squared error} (\textsc{Rmse}),
the \textsc{Mae} is not especially sensitive to outliers
and has a more natural interpretation \cite{mae}.


\subsection{Wasserstein distance} \label{sec:unfolding:metrics:wd}
The two previous metrics were based on single predictions for each event.
They disregard both
  the confidences of the predictions,
    considering only the prediction with the highest confidence,
  and the spectrum,
    which results from summing of the confidences over all events.

In contrast,
% NOTE: The cited paper calls it "Earth Mover's Distance", but I want the citation here for consistency.
the \emph{Wasserstein distance} \cite{wd}
compares the unfolded spectrum to the true spectrum.
%
It is also known as \emph{earth mover's distance} (\textsc{Emd}),
  hinting at the analogy of moving earth to transform one distribution into another,
    where the cost is given as the product of the distance and the amount of earth moved.
      % essentially the physical work.

Mathematically, the Wasserstein distance (of first kind) can be defined as
\begin{equation}
  \text{\textsc{Wd}}(\mathbf{p}, \mathbf{q}) = \inf_{\pi \in \Pi(\mathbf{p}, \mathbf{q})} \int_{\mathbb{R}^2} |x - y| \, \mathrm{d}\pi(x, y)
\end{equation}
where
  $\mathbf{p}$ and $\mathbf{q}$ are the probability distributions subject to comparison,
  $\Pi(\mathbf{p}, \mathbf{q})$ is the set of all probability distributions on $\mathbb{R}^2$,
  and $\pi$ is a probability distribution on $\mathbb{R}^2$.
% The Wasserstein distance is zero if and only if $\mathbf{p} = \mathbf{q}$.

  \section{Hyperparameters}
\subsection{Batch size}
The \emph{batch size} determines the number of events used for each training step.
While larger batch sizes increase the speed of training
on optimized hardware,
the quality of the local minima can be negatively affected.
% TODO: citation needed


\subsection{Adaptive step size: $J$-factor}
\begin{figure}
  \centering
  % TODO: correct dimensions
  \includegraphics[scale=1]{content/plots/halftime/wd_per_J_factor.pdf}
  \caption{…}
  \label{fig:hyperparameter:J_factor}
\end{figure}

\subsection{Number of epochs}

  \section{Bootstrapping and results}

\begin{figure}
    \centering
    \includegraphics[scale=1]{content/plots/bootstrap:spectrum.pdf}
    \caption{
        Energy spectrum and relative deviations of the bootstrap.
        TODO: Placeholder data.
    }
    \label{fig:bootstrap:spectrum}
\end{figure}


\todo{Show/discuss individual events!}
\begin{figure}
    \centering
    % TODO: correct dimensions
    \includegraphics[width=\textwidth]{content/plots/halftime/single_events.pdf}
    \caption{
        Confidence distributions of individual events.
        For each true class,
        one random event is selected from the test set,
        and the confidence distribution of the model is shown.
        }
    % \label{fig:individual_events}
\end{figure}

  \section{Bias}
\textsc{Dsea} is intended to eliminate the bias introduced by the Monte-Carlo-simulated energy spectrum.

In order to test
whether the bias is indeed eliminated,
the model is trained on a \emph{stratified} dataset,
% TODO: explain that dataset: number of events etc.
where each bin contains an equal number of events.
The model is then evaluated on the unmodified dataset.
% TODO: Make clear that train/test data don't overlap
% TODO: This is the opposite of what I did earlier. But it matches what Samuel does. Both should be fine.

The results are shown in \autoref{fig:bias_comparison}.
As can be seen,
only a small bias remains:
The model adapts to the unseen distribution of the test data.
… in comparison to \autoref{fig:bootstrap:spectrum}.

In the work by \citeauthor{dsea_samuel},
relative deviations of more than \SI{1500}{\percent} are observed \cite{dsea_samuel}.
% Aber er schiebt's auf die geringe Accuracy…


\begin{figure}
    \centering
    \includegraphics[width=\textwidth]{content/plots/halftime/fixweighting_spectra_comparison.jpg}
    \caption{TODO.}
    \label{fig:bias_comparison}
\end{figure}

% TODO: Plot example of a biased reconstruction?

  \section{(Comparison to conventional Neural Networks)}
  % \blindtext[3]
  \section{(Comparison to LogisticAT)}
  % \blindtext[3]
\chapter{Summary and Outlook} \label{sec:summary}
% should be 1 page

% █ What did I achieve?
It has been shown that
the combination of neural networks, ordinality and \dsea{}
  can be successfully applied to
  the problem of neutrino energy spectrum estimation
  with \icecube{} data.
This was enabled by
  adding support for
    sample weights
    and confidences
  to \corn{}.

% █ comparisons
In contrast to previous works \cite{dsea_jan, dsea_samuel},
  adaptive step sizes \cite{dsea_mirko} are used in \dsea{},
    % This isn't exactly verified in this thesis…
    drastically reducing the number of iterations required for convergence.
Furthermore,
  under- and overflow bins are added
    to allow for the application to real data.

% With regard to the resulting spectra,
The new method is not unambiguously superior to the previous ones:
% 1. confidence distributions → Samuel
  The randomly selected confidence distributions of a common neural network using softmax \cite{dsea_samuel}
    are of comparable quality to those obtained in this work (see \autoref{fig:bootstrap:single_events}),
      even though ordinality is disregarded in the former case.
  % Compared to the confidence distributions of \emph{LogisticAT} \cite{dsea_jan}, …
%
% 2. accuracy → Samuel
The accuracy …
% 3. Wasserstein distance → Jan
The Wasserstein distance …
% The resulting spectra are not unambiguously better

A strict comparison is not possible,
  as this work introduces under-/overflow bins.

Compared to \cite{dsea_jan},
  better \hyperref[sec:unfolding:metrics:wd]{Wasserstein distances} are achieved
    (\num{0.00879} vs. TODO),
    but using \num{10} instead of \num{12} bins.
On the other hand,
the probability distributions of single events are not strictly unimodal.


% █ future work
There is still a multitude of ways in which \dsea{} and the application thereof could be improved.
\begin{itemize}
  \item Regularization / smoothing
  \item Optimization of other hyperparameters, such as the shape of the neural network
  \item Different network architecture, e.g. with convolutional layers
  \item More data
\end{itemize}

A benefit of neural networks is their good performance on \enquote{raw} data.
However,
the present work has relied on a preprocessing step % aka feature engineering
  – the computation of features such as angles and cone fits –
that is not necessarily required.
Given an adequate neural network architecture and sufficient computing power,
it might be possible to train a neural network directly on the raw data.

TODO: Link to the CNN paper that already exists \cite{minh2021gnn}.

% \section{(Comparison to Conventional Neural Networks)}
% \section{(Comparison to LogisticAT)}



\appendix
% Hier beginnt der Anhang, nummeriert in lateinischen Buchstaben
\chapter{Appendix}
\blindtext

\section{Dataset}
\begin{figure}
    \centering
    \includegraphics[scale=1]{content/plots/dataset:raw:histogram.pdf}
    \caption{Histogram of the full, untouched Monte-Carlo dataset using 30 bins.}
    \label{fig:dataset:raw:histogram}
\end{figure}

\section{Links etc.}
\begin{description}
    \item[Dataset in the chair's \texttt{POOL} file system] \texttt{/net/big-tank/POOL/users/lkardum/new\_mc\_binning.csv} (\SI{14.6}{\giga\byte})
    % https://git.e5.physik.tu-dortmund.de/shaefs/bachelor_thesis/
    % https://git.e5.physik.tu-dortmund.de/nweitkemper/ba % TODO: move master → main
\end{description}
% TODO: Link to GitHub / GitLab / code on Phobos


\backmatter
\printbibliography

\chapter*{Acknowledgements}

Thanks to
  Prof.~Dr.~Dr.~Wolfgang~Rhode and
  Prof.~Dr.~Johannes~Albrecht
  for taking the time to review my thesis
  and for being open to questions.

I would also like to thank
my supervisors
  Karolin~Hymon and
  Leonora~Kardum
as well as
  Tim~Ruhe and
  Mirko~Bunse
for their support and guidance throughout this project.

Furthermore,
I am grateful to Samuel~Haefs,
  who worked on his Bachelor's thesis in parallel with me,
for the inspiring discussions
and for providing a baseline for the implementation of neural networks in \dsea{}.

I appreciate the chair~E5b providing
  the technical infrastructure
  and helpful feedback
    regarding my half-time talk.

% NOTE: Nöthe → Linhoff
Thanks to Dr.~Maximilian~Linhoff for providing a \href{https://github.com/maxnoe/tudothesis}{\LaTeX{} template} for this thesis.

A special thanks to my family for their continued support and encouragement.


\cleardoublepage
\includepdf[pages=-]{content/affidavit.pdf}
% TODO ↓
% \begin{foreignlanguage}{german}
  % \input{content/eid_versicherung.tex}
% \end{foreignlanguage}
\end{document}
