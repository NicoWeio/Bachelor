\section{Dortmund Spectrum Estimation Algorithm} \label{sec:dsea:dsea}
The \textbf{D}ortmund \textbf{S}pectrum \textbf{E}stimation \textbf{A}lgorithm,
  \dsea{} in short, % TODO: language
is a method for solving the previously stated deconvolution problem.
It was introduced by \citeauthor{dsea_tim} in \citeyear{dsea_tim} \cite{dsea_tim}
and improved by \citeauthor{dsea_mirko} in \citeyear{dsea_mirko} \cite{dsea_mirko}.
The improved version
  – which this thesis is based on –
is also referred to as \dseaplus{}.
%
A formal definition of the \dseaplus{} algorithm is given in \autoref{sec:alg:dseaplus}.


\subsection{Deconvolution as a classification task} % identical to Jan's paper
\dsea{} makes use of arbitrary classifiers to solve the deconvolution problem.
This is an advantage,
  as the choice of a classifier can be tailored to the specific problem at hand.
% ORIG: The novel algorithmic framework Dsea is unique among these algorithms,
% because it translates the deconvolution problem into a multinomial classification task,
% thus opening deconvolution to the field of machine learning and the rapid advances being made in that field.
Contrary to other algorithms like
  \textsc{Truee} / \textsc{Run} \cite{milke2013},
  \textsc{Ibu} \cite{dagostini1995, dagostini2010},
no restrictions on the input data are imposed.

Furthermore,
\dsea{} transparently provides the contributions of individual observations to the deconvolved spectrum.
This not only gives deeper insight into the performance of the algorithm,
but also allows for time-dependent deconvolution,
  for example \cite{dsea_mirko}. % Should I really cite this?
In order to use a classifier,
  the deconvolution problem is reformulated as a multinomial classification task.
The spectrum is discretized.


% \subsection{Iterative reproduction of the target density and reweighting of training samples} % identical to Jan's paper
\subsection{Iterative procedure}
\subsubsection{Initialization}
Since no prior knowledge about the true spectrum is available,
  the initial spectrum is chosen to be uniform:
\begin{equation}
  \hat f_j^{(0)} = \frac{1}{N} \quad \forall j \, .
\end{equation}
% TODO: Explain variables here, not below?
The initial weights are then determined as in \autoref{eqn:dsea:weighting}.

% - The resulting bins are used as classes.

\subsubsection{Iteration}
First,
% on the training data,
the classifier is trained to predict the class of a single sample,
where each sample is weighted according to its true class.

Second,
% on the test/real data,
the classifier is used to predict the class of each sample.

Third,
the predicted classes are used to
  % reweight the training samples
  get an updated estimate of the spectrum
  and,
    consequently,
  updated weights for the next iteration.
%
The updated spectrum is determined by the sum of the confidences of all events.
% TODO: j → q for consistency with CORN?
For each energy bin with index $j$,
this can be written as
\begin{equation}
  \hat f_j = \frac{1}{N} \sum_{i=1}^N \hat c_{i,j} \, ,
\end{equation}
where $N$ is the number of events
and $\hat c_{i,j}$ is the confidence of event $i$ for class $j$.
The factor $\sfrac{1}{N}$ is introduced to normalize the spectrum
to a true probability density distribution.

- The resulting spectrum is the preliminary deconvolution result.

- The weights of the training samples are updated according to the preliminary deconvolution result.
In \dseaplus{}, the reconstructed spectrum is divided by the training spectrum. % → fixweighting
This is needed to mitigate the impact of the training spectrum on the deconvolution result.
A more detailed reasoning is given in \cite{dsea_mirko}.

\begin{equation}
  \label{eqn:dsea:weighting}
  w_j^{(k+1)} = \frac{\hat f_j^{(k)}}{f_j^\text{train}} \, ,
\end{equation}
where $w_j^{(k+1)}$ is the weight applied to training samples with true bin $j$ in iteration~$k+1$,
$\hat f_j^{(k)}$ is
  the value of the $j$-th bin in
  the current deconvolution result, % Oxford comma
and $f_j^\text{train}$ is
  the value of the $j$-th bin in
  the training spectrum.

- The classifier is retrained with the updated weights.

\subsubsection{Result}
The final deconvolution result is the spectrum obtained after the last iteration.
…?


\subsection{Step size functions} \label{sec:dsea:dsea:stepsize}
\dseaplus{} introduces the concept of a step size $\alpha$,
which allows the user to control the speed of convergence,
which in turn has a significant impact on the quality of the result.

While the original \dsea{} algorithm uses a fixed step size of $\alpha = 1$,
\dseaplus{} allows arbitrary constants $\alpha > 0$
or functions of the iteration number $k$.
Commonly used step size functions include
multiplicative decay
  $\alpha^{(k)} = k^{\eta - 1}$
and exponential decay
  $\alpha^{(k)} = \eta^{(k - 1)}$,
each with a \emph{decay rate} $0 < \eta < 1$.

These decaying step sizes ensure that the algorithm converges,
decreasing the importance of the maximum number of iterations $K$,
while enabling the use of a $\chi^2$ stopping criterion,
  expressed by a minimum $\chi^2$ distance $\epsilon$ between iterations.
% NOTE: This had already been suggested in the original DSEA paper (→ dsea_tim).


% TODO: adaptive step size function; see Mirko's chapter 3.3
