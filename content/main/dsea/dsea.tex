\section{DSEA} % TODO: Avoid duplicate section names
The \textbf{D}ortmund \textbf{S}pectrum \textbf{E}stimation \textbf{A}lgorithm,
  short \dsea,
is a method for solving the previously stated deconvolution problem.
It was introduced by \citeauthor{dsea_tim} in \citeyear{dsea_tim}
and improved by \citeauthor{dsea_mirko} in \citeyear{dsea_mirko}.
The improved version
  – which this thesis is based on –
is also referred to as \dseaplus{}.

A formal definition of the \dseaplus{} algorithm is given in \autoref{sec:alg:dseaplus}.


\subsection{Deconvolution as a classification task} % identical to Jan's paper
\blindtext[2]


\subsection{Iterative reproduction of the target density and reweighting of training samples} % identical to Jan's paper
\blindtext[4]


\subsection{Step size functions} \label{sec:dsea:dsea:stepsize}
\dseaplus{} introduces the concept of a step size $\alpha$,
which allows the user to control the speed of convergence,
which in turn has a significant impact on the quality of the result.

While the original \dsea{} algorithm uses a fixed step size of $\alpha = 1$,
\dseaplus{} allows arbitrary constants $\alpha > 0$
or functions of the iteration number $k$.
Commonly used step size functions include
multiplicative decay
  $\alpha^{(k)} = k^{\eta - 1}$
and exponential decay
  $\alpha^{(k)} = \eta^{(k - 1)}$,
each with a \emph{decay rate} $0 < \eta < 1$.

These decaying step sizes ensure that the algorithm converges,
decreasing the importance of the maximum number of iterations $K$,
while enabling the use of a $\chi^2$ stopping criterion,
  expressed by a minimum $\chi^2$ distance $\epsilon$ between iterations,
  which had already been suggested in the original \dsea{} paper \cite{dsea_tim}.


% TODO: adaptive step size function; see Mirko's chapter 3.3
