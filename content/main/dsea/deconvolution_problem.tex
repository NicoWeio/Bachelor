\section{The deconvolution problem} % / Inverse problems
% \subsection{Inverse problems}
% TODO: Make clear the difference between inverse problems and deconvolution
Inverse problems are omnipresent in modern physics.
They occur whenever a physical quantity is measured indirectly.
For example,
the intensity of light can measured by a photodetector,
  which converts the light into an electrical signal,
    thus measuring the intensity \emph{indirectly}.
However,
this conversion is not perfect:
The detector has limited acceptance and resolution,
and the signal is subject to noise.
% forward / inverse problem
For neutrino astronomy,
  energy measurements are even more indirect,
    going from neutrinos to leptons to Cherenkov light to electrical signals.
%
The deconvolution problem is to reconstruct the physical quantity of interest
  from the indirect measurements.

% - Fredholm integral equation
Mathematically,
the deconvolution problem can be formulated as an integral equation.
%
The notation is based on \citeauthor{deconvolution_blobel} \cite{deconvolution_blobel}.
% In particular,
%   the meaning of $x$ and $y$ is swapped in the other chapters.
A set of single physical quantities $x$
  with $a \leq x \leq b$ % TODO: Style
  (such as the energy of a neutrino)
can be interpreted as a sample from a probability density $f(x)$.
%
Given
  the true distribution of a physical quantity $f(x)$,
  the measured distribution $g(x)$,
  and a \emph{response function} $A(x, y)$,
  % TODO: mention \Omega
% the deconvolution problem is
%   to find $f(x)$
%     such that $g(x) = A(x, y) f(y)$.
the deconvolution problem is given by
the Fredholm integral equation of the first kind: \cite{deconvolution_blobel}
\begin{equation}
  \label{eq:deconvolution_problem:fredholm}
  % \int_{\mathbb{R}^d} \phi(x) \, \varphi(x - y) \, \mathrm{d}x = \varphi(y) \, .
  \int_a^b A(x, y) f(x) \, \mathrm{d}x = g(y) \, .
\end{equation}


\subsection{Discretization}
In the context of physical measurements,
the integral equation is discretized
  to account for the finite number of samples.
The continuous distribution functions $f(x)$ and $g(x)$ are replaced by vectors $\vec{f}$ and $\vec{g}$,
and the kernel (response function) $A(x, y)$ by a \emph{transfer matrix} $\symbf{A}$.
%
% Disregarding the bias term, % TODO: if there was one in the first place…
The discretized deconvolution problem is then given by
\begin{equation}
  \label{eq:deconvolution_problem:discretized}
  \symbf{A} \vec{f} = \vec{g} \, .
\end{equation}

In practice,
the transfer matrix $A(x, y)$ can be approximated
by Monte Carlo simulations of the detector,
  where both the true and the measured quantities are known.
Given an actual set of measurements $\vec{g}$,
the deconvolution problem can then be solved
by inverting the transfer matrix $\symbf{A}$:
\begin{equation}
  \label{eq:deconvolution_problem:discretized:inverse}
  \vec{f} = \symbf{A}^{-1} \vec{g} \, .
\end{equation}

However,
% as is common with inverse problems,
the matrix inversion operation is usually \emph{ill-conditioned},
  leading to numerical instabilities
  and oscillations in the solution.

% It is \emph{ill-conditioned}:
%   the solution can change drastically
%     when the measurements are slightly perturbed.
% % COPILOT: This is because the solution is not unique:
% %   there are many different physical quantities
% %     that can produce the same measurements.

One approach to overcome this problem
is \emph{regularization}.
% Regularization is a technique
% to make the deconvolution problem well-posed
% by adding a penalty term to the objective function.
Regularization allows for better results
  at the cost of introducing additional assumptions and parameters
    (\emph{Bias-variance trade-off}).
A common regularization technique
  is to penalize the second derivative of the solution.
    % which is known as \emph{Tikhonov regularization}. % TODO: Is it really?
It incentivizes the solution to be smooth,
  which is often a reasonable assumption for distributions of physical quantities.

