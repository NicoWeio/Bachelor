\section{The Deconvolution Problem} % / Inverse problems
% \subsection{Inverse Problems}
Inverse problems are omnipresent in modern physics.
They occur whenever a physical quantity is measured indirectly.
For example,
the intensity of light can be measured by a photodetector,
  which converts the light into an electrical signal.
However,
this conversion is not perfect:
A real detector has limited acceptance and resolution
and the signal is subject to noise.
% forward / inverse problem
% For neutrino astronomy,
%   energy measurements are even [more indirect],
%     going from neutrinos to leptons to Cherenkov light to electrical signals.
%
The deconvolution problem
  – as a special case of inverse problems –
  is to reconstruct
    the distribution of the physical quantity of interest
    from the indirect measurements.

% █ Fredholm integral equation
% NOTE: The notation is based on Blobel (→ deconvolution_blobel).
% In particular, the meaning of $x$ and $y$ is swapped in the other chapters.
% NOTE: There has been some discussion about what f(x) is.
% - Blobel says that f is a probability density function (and therefore, f(x) is a probability density).
%   I would have to integrate over an energy range to get a flux.
% - Leonora says that f(x) is a flux or an event spectrum (and therefore a physical quantity itself).
% - IMHO Leonora's interpretation is a special case.
%   Leonora, on the other hand, considers Blobel's interpretation a special case.
Mathematically,
a set of single physical quantities $x$
  limited to an arbitrary range $a \leq x \leq b$
    (such as the energy of a neutrino)
can be interpreted as samples from an event spectrum $f(x)$.
%
Given
  % this true probability density,
  the measured distribution $g(y)$
  and a \emph{response function} $A(x, y)$,
    which describes the detector,
the deconvolution problem is
  to find a distribution $f(x)$ that satisfies
  the Fredholm integral equation of the first kind \cite{deconvolution_fredholm}:
\begin{equation}
  % NOTE: Blobel adds a bias term, but it's not part of the Fredholm equation.
  \label{eq:deconvolution_problem:fredholm}
  \int_a^b A(x, y) f(x) \, \mathrm{d}x = g(y) \, .
\end{equation}


\subsection{Discretization} \label{sec:dsea:deconvolution_problem:discretization}
In the context of physical measurements,
the integral equation is discretized
  to account for the finite number of samples.
The continuous distribution functions $f(x)$ and $g(y)$ are replaced by vectors $\vec{f}$ and $\vec{g}$,
and the kernel (response function) $A(x, y)$ by a \emph{transfer matrix} $\symbf{A}$.
%
The discretized deconvolution problem is then given by
\begin{equation}
  \label{eq:deconvolution_problem:discretized}
  \symbf{A} \vec{f} = \vec{g} \, .
\end{equation}

In practice,
the transfer matrix can be approximated
by Monte Carlo simulations of the detector,
  where both the true and the measured quantities are known.
Given an actual set of measurements $\vec{g}$,
the deconvolution problem can then be solved
by inverting the transfer matrix:
\begin{equation}
  \label{eq:deconvolution_problem:discretized:inverse}
  \vec{f} = \symbf{A}^{-1} \vec{g} \, .
\end{equation}

However,
as is common with inverse problems,
the matrix is usually \emph{ill-conditioned},
    % since the underlying inverse problem is ill-posed,
  leading to numerical instabilities
  and oscillations in the solution.

% It is \emph{ill-conditioned}:
%   the solution can change drastically
%     when the measurements are slightly perturbed.
% % COPILOT: This is because the solution is not unique:
% %   there are many different physical quantities
% %     that can produce the same measurements.

\phantomsection \label{sec:dsea:deconvolution_problem:regularization}
One approach to overcome this problem
is \emph{regularization}.
% Regularization is a technique
% to make the deconvolution problem well-posed
% by adding a penalty term to the objective function.
Regularization allows for better results
  at the cost of introducing additional a priori assumptions and parameters
    (\emph{bias-variance trade-off}) \cite{bias_variance_tradeoff}. % aka smoothing/sharpening trade-off
A common regularization technique is
  to penalize the second derivative of the solution \cite{deconvolution_starck},
    which is known as \emph{Tikhonov regularization} \cite{tikhonov_regularization}.
This incentivizes the solution to be smooth,
  which is often a reasonable assumption for distributions of physical quantities.

