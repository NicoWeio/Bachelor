\section{CORN}
% Conditional Ordinal Regression for Neural Networks
\emph{CORN} (\textbf{C}onditional \textbf{O}rdinal \textbf{R}egression for \textbf{N}eural Networks) \cite{corn}
is a novel (?) framework for ordinal classification in neural networks.
It is based on the ideas of separation into binary subtasks and conditional probabilities.

% TODO: Explain CORN

% TODO: Reduce nesting ↓
\subsection{Getting probabilities from CORN} % OR: Converting threshold to per-class probabilities
As explained before,
CORN returns conditional (?) probabilities based on the binary classification subtasks it uses.
In contrast,
the DSEA algorithm is strongly coupled to the idea of per-class probabilities (confidences).
Therefore, a conversion is necessary.
Under the assumption that all energies belong to one of the bins,
it can be shown that the conditional probabilities can be converted to per-class probabilities.

As an example, given four classes $0, 1, 2, 3$,
the assuption can be written as $y \in \{0, 1, 2, 3\}$.
The conditional probabilities are then given by
\begin{align*}
  % NOTE: P[y>(-1)] = 1 and P[y>3] = 0 :)
  P[y=0] &= 1 - P[y>0] \\
  P[y=1] &= P[y>1] - P[y>1] \\
  P[y=2] &= P[y>1] - P[y>2] \\
  P[y=3] &= P[y>2]
\end{align*}

A more detailed explanation is given in \autoref{sec:appendix:corn_probas}.
