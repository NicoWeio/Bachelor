\section{On Nominal and Ordinal Data}
% █ Nominal data
Nominal data can be thought of as a set of distinct categories
  (\emph{classes})
with no inherent ordering.
The most common classifiers,
  such as
    neural networks with softmax output \cite{dsea_samuel},
    and random forests \cite{hymon2021seasonal},
    % …
treat data as nominal.
%
% █ Ordinal data
Ordinal data, on the other hand,
has an inherent ordering.
The distance between two categories is not necessarily the same. % … not considered.
Instead of classes,
  the different categories are commonly referred to as \emph{ranks}.
There are classifiers that can handle ordinal data,
such as
  \emph{LogisticAT} \cite{logisticat, dsea_jan}
  % COPILOT \emph{Ordinal Regression Forests} \cite{ordinal_regression_forests},
  and \corn{} (see \autoref{sec:ordinal:corn}).

% █ Application
In the context of the problem at hand,
  the neutrino energies are ordinal,
    since they are discretized into bins,
      while the ordering by energy remains intact.

A classifier can then either
  treat the data as nominal,
    disregarding the ordering,
  or as ordinal,
    taking the ordering into account.
The latter is more appropriate,
  since the ordering is known to be important. % TODO
% TODO: Hier evtl. auf "single events" verweisen


% This means that the classifier is trained to predict
% which of the $I$ energy bins a sample belongs to
%   without taking into account …. % TODO

    % the relative abundance of the classes.
% Given small enough bins,
% a correlation between neighboring bins is expected.


% █ Metric data → why not?
% It shall be noted that
% the discretized neutrino energies are not metric in the strictest sense,
%   since the under-/overflow bins are of different width than the others.

% due to the discretization necessary for \dsea{},
%   the neutrino energies are no longer metric data.
% This is acceptable
% because all bins
%   – apart from the under-/overflow bins –
%  are of equal width.
