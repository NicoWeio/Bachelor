\section{On Nominal and Ordinal Data}
% █ Nominal data
Nominal data can be thought of as a set of distinct categories
  (\emph{classes})
with no inherent ordering.
The most common classifiers,
  such as
    neural networks with softmax output \cite{dsea_samuel},
    and random forests \cite{hymon2021seasonal},
    % …
treat data as nominal.
%
% █ Ordinal data
Ordinal data, on the other hand,
has an inherent ordering.
The distance between two categories is not necessarily the same. % … not considered.
Instead of classes,
  the different categories are commonly referred to as \emph{ranks}.
There are classifiers that can handle ordinal data,
such as
  \emph{LogisticAT} \cite{logisticat, dsea_jan}
  % COPILOT \emph{Ordinal Regression Forests} \cite{ordinal_regression_forests},
  and \corn{} (see \autoref{sec:ordinal:corn}).

% █ Application
In the context of the problem at hand,
  the neutrino energies are ordinal,
    since they are discretized into bins,
      while the ordering by energy remains intact.
%
A classifier can then either
  treat the data as nominal,
    disregarding the ordering,
  or as ordinal,
    taking the ordering into account.
Disregarding the ordering
is especially problematic for the reconstruction of
  single events
  or spectra that depend on an additional parameter
    (such as the time \cite{hymon2021seasonal}),
because in both cases,
  not only the unfolded spectrum
    (as a normalized sum of confidence distributions),
  but also the confidence distribution of single events
  is considered.
With nominal classification,
  there is no incentive for the classifier
    to return an unimodal confidence distribution.
An event could therefore yield
  high confidences for both very low and very high energies,
  which is not physically plausible.


% NOTE: There's also metric data, which means that the distance between two categories is quantifiable.
% One could argue that – since our inner bins are of equal width – the energy is even metric.
