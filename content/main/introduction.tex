\chapter{Introduction}
% █ astroparticle physics
Astroparticle physics is a relatively new field of physics
  that explores the universe
    with messenger particles.
They carry information about
    the production processes
    and the environment they were created in,
  which can be obtained upon their detection.
% This way, AGN, supernovae, and other astrophysical phenomena can be studied.
% █ neutrino astronomy
Neutrinos are especially valuable messenger particles
  because their propagation path is not affected by electromagnetic fields,
  and they can propagate long distances without interacting with matter.
% Therefore,
%   both their source and their energy can be determined
%   by measuring the direction and energy of
%   decay products that they produce in the detector on Earth.

% The most common messenger particles are neutrinos,
% Many high-energy processes are assumed to produce neutrinos.

% █ IceCube
\icecube{} is a detector for such neutrinos,
  which is located at the South Pole.
The Antarctic ice is used as detector material
  and is sensitive to high-energy neutrinos. % COULDDO: imprecise
One goal of analyses with \icecube{} is
  to obtain a neutrino energy spectrum.

% █ DSEA
The \acf{DSEAPLUS} \cite{dsea_unification}
is a method to reconstruct the (discretized) neutrino energy spectrum
  from measured data of \icecube{}.
While classifiers such as random forests
have been successfully applied to \icecube{} data in \dsea{} \cite{hymon2021seasonal},
the possibilities of neural networks and deep learning are still largely unexplored. % Naja…
%
\citeauthor{dsea_samuel} \cite{dsea_samuel}
combined \dsea{} with a neural network for the first time.
%
\citeauthor{dsea_jan} \cite{dsea_jan},
  on the other hand,
focused on the ordinality of the discretized energy spectrum
  by employing \emph{LogisticAT} \cite{logisticat},
    a classifier
      that considers
        the ordering of energy bins
      in its loss function.
%
% COULDDO: Differentiate from abstract ↓
This thesis aims to combine the advantages of both approaches:
  the flexibility of neural networks
  and the potential improvements in physical plausibility
    due to respecting ordinality.
% █ CORN
The \acf{CORN} framework \cite{corn}
  allows neural networks to respect ordinality
  by using a special
    loss function
    and activation function
      in the output layer.
It is adapted to work with \dsea{}
  by converting from conditional confidences to per-class confidences
  and by adding support for sample weights.
This combination of \acf{CORN} and \dsea{}
is then optimized and evaluated
  on simulated data from \icecube{}.

% █ structure
This paper is organized as follows:
\Cref{sec:neutrino_astronomy} briefly introduces \icecube{} and important concepts of neutrino astronomy.
\Cref{sec:dsea} describes the deconvolution problem and the \dsea{} approach to solving it.
\Cref{sec:ordinal} explains the need for ordinal classification and introduces the \ac{CORN} framework.
\Cref{sec:unfolding} describes the setup for hyperparameter searches and evaluates the performance of the optimized model.
\Cref{sec:summary} summarizes the thesis and discusses future work.
