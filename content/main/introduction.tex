\chapter{Introduction}
% █ astroparticle physics
% █ neutrino astronomy / IceCube
% █ DSEA
While [conventional] machine learning techniques
  such as Random Forests
have been successfully applied to \icecube{} data in \dsea{},
the possibilities of neural networks and deep learning are still largely unexplored. % Naja…
%
Previous works have either
  focused on respecting ordinality… \cite{dsea_jan}
  or on using a neural network as a classifier \cite{dsea_samuel}.
This thesis aims to combine the advantages of both approaches:
  the flexibility of neural networks
  % and the ability to handle ordinality.
  and the physical plausibility of respecting ordinality.
% █ CORN
This is done by adapting the
\corn{} (\textbf{C}onditional \textbf{O}rdinal \textbf{R}egression for \textbf{N}eural Networks) \cite{corn} framework
to \dsea{}
and applying it to \icecube{} Monte Carlo data.

% █ structure
\Cref{sec:neutrino_astronomy} briefly introduces \icecube{} and important concepts of neutrino astronomy.
\Cref{sec:dsea} describes the deconvolution problem and the \dsea{} approach to solving it.
\Cref{sec:ordinal} [rationalizes/justifies] the need for ordinal classification and introduces the \corn{} framework.
\Cref{sec:unfolding} describes the [experimental] setup and the results of the [experiments].
\Cref{sec:summary} concludes the thesis and discusses future work.

\blindtext[2]
