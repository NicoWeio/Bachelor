\chapter{Introduction}
% █ astroparticle physics
Astroparticle physics is a relatively new field of physics
  that explores the universe
    with the help of messenger particles.
They carry information about
    the processes that produced them
    and the environment they were created in,
  which can be obtained upon their detection.
% This way, AGN, supernovae, and other astrophysical phenomena can be studied.
% █ neutrino astronomy
Neutrinos are especially valuable messenger particles
  because their propagation path is not affected by electromagnetic fields,
  and they can travel long distances without interacting with matter.
% Therefore,
%   both their source and their energy can be determined
%   by measuring the direction and energy of
%   decay products that they produce in the detector on Earth.

% The most common messenger particles are neutrinos,
% Many high-energy processes are assumed to produce neutrinos.

% █ IceCube
\icecube{} is such a detector,
  which is located at the South Pole.
% ↓ move to icecube.tex?
It makes use of the Antarctic ice as detector material
  and is sensitive to high-energy neutrinos
    in the approximate range of \si{\tera\electronvolt} to \si{\peta\electronvolt} \cite{icecube_aartsen}.
      % \SI{E7}{\electronvolt} to \SI{E21}{\electronvolt}. % source?
%
One goal of analyses with \icecube{} is
  to obtain a neutrino energy spectrum.

% █ DSEA
The \textbf{D}ortmund \textbf{S}pectrum \textbf{E}stimation \textbf{A}lgorithm (\dsea{}) \cite{dsea_unification}
is a method to reconstruct the neutrino energy spectrum
  from the measured data of \icecube{}.
While [conventional] classifiers % TODO
  such as random forests
have been successfully applied to \icecube{} data in \dsea{},
the possibilities of neural networks and deep learning are still largely unexplored. % Naja…
%
Previous works
% that perform unfolding…
have focused either
  on respecting ordinality \cite{dsea_jan} % TODO: Can I explain ordinality here?
  or on using a neural network as a classifier \cite{dsea_samuel}.
This thesis aims to combine the advantages of both approaches:
  the flexibility of neural networks
  and the potential improvements in physical plausibility
    due to respecting ordinality.
% █ CORN
This is done by adapting the
\corn{} (\textbf{C}onditional \textbf{O}rdinal \textbf{R}egression for \textbf{N}eural Networks) framework \cite{corn}
to \dsea{}.
The proposed method is then optimized and evaluated
  on simulated data from \icecube{}.

% █ structure
\Cref{sec:neutrino_astronomy} briefly introduces \icecube{} and important concepts of neutrino astronomy.
\Cref{sec:dsea} describes the deconvolution problem and the \dsea{} approach to solving it.
\Cref{sec:ordinal} [rationalizes/justifies] the need for ordinal classification and introduces the \corn{} framework.
\Cref{sec:unfolding} describes the setup for hyperparameter searches and evaluates the performance of the resulting model. % TODO
\Cref{sec:summary} concludes the thesis and discusses future work.
