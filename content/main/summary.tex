\chapter{Summary and outlook} \label{sec:summary}
% should be 1 page

% █ What did I achieve?
It has been shown that
the combination of neural networks, ordinality and \dsea{}
  can be successfully applied to
  the problem of neutrino energy spectrum estimation
  with \icecube{} data.
This was enabled by
  adding support for
    sample weights
    and confidences
  to \corn{}.

% █ comparisons
In contrast to previous works \cite{dsea_jan, dsea_samuel},
  adaptive step sizes \cite{dsea_mirko} are used in \dsea{},
    % This isn't exactly verified in this thesis…
    drastically reducing the number of iterations required for convergence.
Furthermore,
  under- and overflow bins are added
    to allow for the application to real data.

% With regard to the resulting spectra,
The new method is not unambiguously superior to the previous ones:
% 1. confidence distributions → Samuel
  The randomly selected confidence distributions of a common neural network using softmax \cite{dsea_samuel}
    are of comparable quality to those obtained in this work (see \autoref{fig:bootstrap:single_events}),
      even though ordinality is disregarded in the former case.
  % Compared to the confidence distributions of \emph{LogisticAT} \cite{dsea_jan}, …
%
% 2. accuracy → Samuel
The accuracy …
% 3. Wasserstein distance → Jan
The Wasserstein distance …
% The resulting spectra are not unambiguously better

A strict comparison is not possible,
  as this work introduces under-/overflow bins.

Compared to \cite{dsea_jan},
  better \hyperref[sec:unfolding:metrics:wd]{Wasserstein distances} are achieved
    (\num{0.00879} vs. TODO),
    but using \num{10} instead of \num{12} bins.
On the other hand,
the probability distributions of single events are not strictly unimodal.


% █ future work
There is still a multitude of ways in which \dsea{} and the application thereof could be improved.
\begin{itemize}
  \item Regularization / smoothing
  \item Optimization of other hyperparameters, such as the shape of the neural network
  \item Different network architecture, e.g. with convolutional layers
  \item More data
\end{itemize}

A benefit of neural networks is their good performance on \enquote{raw} data.
However,
the present work has relied on a preprocessing step % aka feature engineering
  – the computation of features such as angles and cone fits –
that is not necessarily required.
Given an adequate neural network architecture and sufficient computing power,
it might be possible to train a neural network directly on the raw data.

TODO: Link to the CNN paper that already exists \cite{minh2021gnn}.

% \section{(Comparison to conventional Neural Networks)}
% \section{(Comparison to LogisticAT)}
