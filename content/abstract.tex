\thispagestyle{plain}

\section*{Abstract}
% █ Background
The deconvolution of energy spectra is a common problem in neutrino astronomy.
As an inverse problem,
  its solution requires special methods.
The \acf{DSEAPLUS} \cite{dsea_unification}
solves the deconvolution problem
  by reinterpreting it as a multinomial classification problem
  % using an arbitrary classifier to estimate the discretized energy spectrum
and eliminates bias from the training data
  by iteratively re-weighting the samples.
%
In the problem at hand,
  discretized neutrino energies are \emph{ordinal} quantities,
    implying that misclassification can be of different severity.
However,
  most classifiers do not respect this property.
% █ Objective
% NOTE: Duplicate of introduction.tex
Previous works
have focused either
  on respecting ordinality \cite{dsea_jan}
  or on using a neural network as a classifier \cite{dsea_samuel}.
This thesis aims to combine the advantages of both approaches:
  the flexibility of neural networks
  and the potential improvements in physical plausibility
    due to respecting ordinality.
% █ Methods
First,
the \ac{CORN} framework \cite{corn} is adapted to work with \dsea{}.
The proposed method is then optimized and evaluated
  on simulated data from \icecube{}.
% █ Results
% COULDDO


\acresetall % reset all acronyms
\section*{Kurzfassung}
\begin{foreignlanguage}{german}
Die Entfaltung von Energiespektren ist ein gängiges Problem in der Neutrinoastronomie.
Da es sich um ein Inversionsproblem handelt,
  erfordert seine Lösung spezielle Methoden.
Der \ac{DSEAPLUS} \cite{dsea_unification}
löst das Entfaltungsproblem,
  indem es dieses als ein mehrklassiges Klassifikationsproblem uminterpretiert
  % using an arbitrary classifier to estimate the discretized energy spectrum
und Bias aus den Trainingsdaten eliminiert,
  indem es die Samples iterativ neu gewichtet.
%
Im vorliegenden Problem
  sind diskretisierte Neutrinoenergien \emph{ordinal},
    was impliziert, dass Fehlklassifikationen unterschiedlich stark sein können.
Allerdings
  respektieren die meisten Klassifizierer diese Eigenschaft nicht.
% █ Objective
% NOTE: Duplicate of introduction.tex
Vorherige Arbeiten
haben sich entweder
  auf die Berücksichtigung der Ordinalität \cite{dsea_jan}
  oder auf die Verwendung eines neuronalen Netzwerks als Klassifizierer \cite{dsea_samuel}
fokussiert.
Diese Arbeit hat zum Ziel,
  die Vorteile beider Ansätze zu kombinieren:
    die Flexibilität von neuronalen Netzwerken
    und die potenzielle Verbesserung der physikalischen Plausibilität
      aufgrund der Berücksichtigung von Ordinalität.
% █ Methods
Zunächst wird \ac{CORN} \cite{corn} angepasst,
  um mit \dsea{} kompatibel zu sein.
Die vorgeschlagene Methode wird dann
  optimiert
  und auf simulierten Daten von \icecube{} evaluiert.
\end{foreignlanguage}

% NOTE: Acronyms are reset again in thesis.tex
