\thispagestyle{plain}

\section*{Abstract}
% - Neutrinos
% - IceCube
% - DSEA
% - CORN
% - THIS THESIS

The deconvolution of energy spectra is a [common] problem in neutrino astronomy.
As an inverse problem,
  it requires special methods to solve.
The \acf{DSEAPLUS} \cite{dsea_unification}
solves the deconvolution problem
  by using a classifier to estimate the discretized energy spectrum
  and iteratively reweighting the training data.
%
In the given problem,
  neutrino energies are \emph{ordinal} quantities,
    which means that the order of the energies is important. % COPILOT
However,
  most classifiers do not respect this property.
%
% NOTE: Duplicate of introduction.tex
Previous works
have focused either
  on respecting ordinality \cite{dsea_jan}
  or on using a neural network as a classifier \cite{dsea_samuel}.
This thesis aims to combine the advantages of both approaches:
  the flexibility of neural networks
  and the potential improvements in physical plausibility
    due to respecting ordinality.
The \ac{CORN} framework,
  which [provides] ordinal neural networks,
is adapted to work with \dsea{}
and optimized and evaluated on simulated data from \icecube{}.



\section*{Kurzfassung}
\begin{foreignlanguage}{german}
Hier steht eine Kurzfassung der Arbeit in deutscher Sprache inklusive der Zusammenfassung der
Ergebnisse.
Zusammen mit der englischen Zusammenfassung muss sie auf diese Seite passen.

\blindtext[1]
\end{foreignlanguage}
